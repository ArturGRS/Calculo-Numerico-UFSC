\documentclass{article}
\usepackage[utf8]{inputenc}
\usepackage{amsmath, amssymb, amsthm}
\usepackage{graphicx, float}
\usepackage{multicol}
\usepackage[brazil]{babel}
\usepackage{pgfplots}
\pgfplotsset{compat=1.18}
\usepackage[letterpaper, top = 1in, bottom = 1.0 in, left = 0.8 in, right = 0.8 in, heightrounded]{geometry}

%%%%%%%%%%%%%%%%%%%%%%%%% Caso haja dúvidas na Symbologia %%%%%%%%%%%%%%%%
% https://detexify.kirelabs.org/classify.html
%%%%%%%%%%%%%%%%%%%%%%%%% Parâmetros de construção %%%%%%%%%%%%%%%%%%%%%%%

\setlength{\parindent}{0pt}
\setlength{\parskip}{0.8em}

\title{\textbf{Repositório de Cálculo Numérico}}
\author{UFSC Joinville - EMB5016 \\ Artur Gemaque}
\date{\today}
%%%%%%%%%%%%%%%%%%%%%%%%%% COMEÇO DO DOCUMENTO %%%%%%%%%%%%%%%%%%%%%%%%%%%
\begin{document}
\maketitle

\begin{abstract}
    Este documento tem a finalidade de repositório atuando como material de reforço 
    para os alunos da disciplina de Cal. Numérico do professor Alexandre Zabot. Nesse sentido, 
    devo informar que os sequintes conteúdos são nada mais do que anotações 
    dos alunos no decorrer das aulas. Agradecimentos ao Maicon que me deu essa ideia!
\end{abstract}

\begin{multicols}{2}
\section{Método da Bissecção}
    O método divide repetidamente pela metade (bisseção) subintervalos de [a, b] e, em cada
    passo, localiza a metade que contém p, sendo p a raiz da equação.

    O método consiste em: 
    \begin{enumerate}
        \item Chute um valor para "a"\ e outro para "b", de maneira que $ f(x) $ seja continua no intervalo [a,b].
        \item Atribua o valor médio de "a"\ e "b"\ a uma variável chamanda de "q".
        \item Certifique-se que "q"\ não é uma raiz.
        \item Verifique o sinal de $ f(a) $ e $ f(q) $.
        \item Caso I. $ f(a) $ e $ f(q) $ contém mesmo sinal $\Longrightarrow $ A raiz está entre "q"\ e "b". 
        \item Substitua "a"\ pelo valor de "q"
        \item Retorne para o iten 2
        \item Caso II. $ f(a) $ e $ f(q) $ contém sinais opostos $\Longrightarrow $ A raiz está entre "a"\ e "q". 
        \item Substitua "b"\ pelo valor de "q"
        \item Retorne para o iten 7
    \end{enumerate}

\section{Iteração de Ponto Fixo}
    Observamos que sempre podemos reescrever uma equação da forma $ f(x) = 0 $, de modo que obtamos a função equivalente $ g(x) = x^* $ 
    . Por exemplo:

    \begin{enumerate}
        \item $ e^x = x + 2 $, podemos reescrever como $ f(x) = 0 $ ou $ f(x) = e^x - x - 2 $
        \item Reescrevemos a função que representa todos os "0" da equação como x = x da sequinte forma $ g(x) = x $ e $ g(x) = e^x - 2 $ 
        \item Dada uma função $ g(x) $, a iteração do ponto fixo consiste em computar a seguinte sequência recursiva: \\ $ x^{(n+1)} = g(x^{(n)}),\  n >= 1, $
    \end{enumerate}



\end{multicols}
\end{document}